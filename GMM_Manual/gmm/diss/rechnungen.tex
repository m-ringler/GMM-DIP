% $Id: rechnungen.tex,v 1.20 2008/05/04 10:41:54 Moritz.Ringler Exp $
% Kein unabh�ngiges LaTeX-Dokument!
% Master-Dokument: diss.tex
\chapter{Theorie der Raman- und Fluoreszenzverst�rkung
in Nanopartikelaggregaten}\label{kap.rechnungen}
%
% Dieses Kapitel liefert aufbauend auf der
% in Kapitel~\ref{kap.grundlagen} eingef�hrten verallgemeinerten Mie-Theorie
% die vollst�ndige theoretische Beschreibung des Nanopartikeldimer-Resonators.
% Zun�chst wird anhand der Streuspektren und des Nahfeldes gezeigt, dass die
% longitudinale Resonanz des Nanopartikeldimers mit dem Abstand der
% Partikel durchgestimmt werden kann bzw. im Umkehrschluss dazu genutzt werden
% kann, den Abstand der Partikel optisch zu messen. Dann wird die im Rahmen
% dieser Arbeit entwickelte
% Theorie der Ramanverst�rkung und der
% Fluoreszenzverst�rkung f�r Molek�le im Resonator vorgestellt.
% Anregungsverst�rkung\footnote{Das Wort "`Verst�rkung"' soll hier und im
% folgenden auch eine m�gliche Unterdr�ckung mit einschlie�en.},
% Emissionsverst�rkung und Energietransfer werden
% mit Hilfe der verallgemeinerten
% Mie-Theorie zun�chst f�r ein Zwei-Niveau-System
% berechnet, und dann werden die Ergebnisse auf ein reales Molek�l �bertragen.

\input{diss/theorie/anregung}
\input{diss/theorie/emission}
\input{diss/theorie/flverst}
\input{diss/theorie/beispiel}
\input{diss/theorie/ramanvst}
\vspace{\baselineskip}

Damit haben wir in diesem Kapitel einen Formalismus entwickelt, der es uns
erlaubt, Fluo\-reszenz- und Ramanverst�rkung f�r reale Molek�le in Aggregaten
beliebig vieler sph�rischer Nanopartikel zu berechnen. Wir werden dieses
Modell in den folgenden beiden Kapiteln einsetzen, um die experimentellen
Ergebnisse dieser Arbeit zu interpretieren. Details zur
Implementierung des Modells als FORTRAN-Programm finden sich im Anhang.



