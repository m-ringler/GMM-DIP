% $Id: definitionen.tex,v 1.33 2008/03/11 09:02:38 Moritz.Ringler Exp $

% Layout und Satz
%---------------------------------------------
%Kopf- und Fusszeile
\pagestyle{scrheadings}
\clearscrheadings
\clearscrplain
\lohead{\headmark}
\ifrcs{
    \lofoot{\RCSId}
    \refoot{\RCSId}
}{}
\lefoot[\pagemark]{\pagemark}
\rofoot[\pagemark]{\pagemark}
\setfootsepline{.4pt} % Ganzunten
%---------------------------------------------
% Rueckverweise im Literaturverzeichnis
%\renewcommand*{\backref}[1]{} % backref wird ausgeschaltet
%\renewcommand*{\backrefalt}[4]{ % backrefalt umdefiniert
%\ifcase #1 %keine Seite
%[nicht zitiert] %
%\or % genau eine Seite
%\mbox{[$\rightarrow$ Seite #2]}
%\else % mehrere Seiten
%\mbox{[$\rightarrow$ Seiten #2]}
%\fi
%}
% Zeichenkette zwischen genau 2 Seiten
%\renewcommand*{\backreftwosep}{ und~}
% Zeichenkette zwischen den beiden letzten von > 2 Seiten
%\renewcommand*{\backreflastsep}{ und~}
%----------------------------------------------
% Captions
%\renewcommand{\captionlabelsep}{\/\ }
%----------------------------------------------
% Sonstige Verzeichnisse, Indices
% Symbolverzeichnis


%Standardindex
% default: Dies ist der Standardindex, es muss kein Bezeichner angegeben werden
% idx: Erweiterung fuer MakeIndex-Eingabedatei
% ind: Erweiterung fuer MakeIndex-Ausgabedatei
% Index: Name fuer den Index
\newindex{default}{idx}{ind}{Index}

%\sloppy

\newcommand{\uri}[1]{\href{#1}{#1}}

\newenvironment{myTitlepage}{ %
    \thispagestyle{empty}
    \addtolength{\oddsidemargin}{0.5cm}
    \begin{titlepage}
    %\addtolength{\textwidth}{1.6cm}
    \begin{minipage}{\textwidth}
    \centering
}{
    \end{minipage}
    %\addtolength{\textwidth}{-1.6cm}
    \enlargethispage{5 \baselineskip}
    \end{titlepage}
    \addtolength{\oddsidemargin}{0.5cm}
}
%\numberwithin{equation}{section}

%\renewcommand{\baselinestretch}{1.5}
\renewcommand{\arraystretch}{0.97}
\newcommand{\foreign}[1]{\textsl{#1}}
\newcommand{\begriff}[1]{{#1}}
\newcommand{\refeqn}[1]{Eq.~\eqref{#1}}
%thesis class benutzt fuer captions \figureshortname,
%wird in ngerman.sty/german.sty nicht umdefiniert.
%\newcommand{\figureshortname}{Abb.}
\newcommand{\reffig}[1]{Fig.~\ref{#1}}
\newcommand{\refsec}[1]{Sec.~\ref{#1}}
%\newcommand{\refapp}[1]{Anhang~\ref{#1}}
%\newcommand{\reftab}[1]{Tabelle~\ref{#1}}
\newcommand{\code}[1]{{\texttt{#1}}}
\newcommand{\bookmark}[3][]{\ifhyperref{\hypertarget{#3}{}\pdfbookmark[#1]{#2}{#3}}{}}
\renewcommand{\dictumwidth}{0.41 \textwidth}

% Meta-Daten
% uncomment exactly one of the following two lines
%\newcommand{\cvsvers}{}
\newcommand{\cvsvers}{{\tiny(Rev. \RCSRevision{}, \RCSDate{})\newline}}

% numerische Werte
\newcommand{\eee}[1]{\ensuremath{10^{#1}}}
\newcommand{\val}[2]{\ensuremath{#1}\text{\,#2}}
\newcommand{\ee}[2]{\ensuremath{#1\times\eee{#2}}}
\newcommand{\valee}[3]{\val{\ee{#1}{#2}}{#3}}
\newcommand{\mikro}[1]{\ensuremath{\text{\micro#1}}}
\newcommand{\ftow}[1]{\ensuremath{2\pi \!\times\!#1}}

% nanoantenna.tex
\newcommand{\wexc}{\ensuremath{\omega_\mathrm{exc}}}
\newcommand{\rexc}{\ensuremath{\gamma_\mathrm{exc}}}
\newcommand{\rexci}{\ensuremath{\gamma_{\mathrm{exc}\, 0}}}
\newcommand{\qe}{\ensuremath{\eta}}
\newcommand{\rr}{\ensuremath{\gamma_\mathrm{r}}}
\newcommand{\rri}{\ensuremath{\gamma_{\mathrm{r}\, 0}}}
\newcommand{\ret}{\ensuremath{\gamma_\mathrm{ET}}}
\newcommand{\rfluo}{\ensuremath{\gamma_\mathrm{Fluo}}}
\newcommand{\rxr}{\ensuremath{\gamma_{\mathrm{nr}}}}

% chem. Elemente
\newcommand{\element}[2]{#1 #2}
\newcommand{\selement}[2]{\ensuremath{\kern0.05em^{#2}}\kern-0.1em#1}

% thermodyn. und statistische Groessen
\newcommand{\ldb}{\ensuremath{\lambda_{\mathrm{dB}}}}
\newcommand{\kb}{\ensuremath{k_\mathrm{B}}}
\newcommand{\visc}{\ensuremath{\tilde{\eta}}}
%\newcommand{\Tkrit}{\ensuremath{T_c}}
%\newcommand{\Tdop}{\ensuremath{T\!_\mathrm{D}}}
%\newcommand{\dm}{\qop{\rho}}
%\newcommand{\dmel}[2]{\ensuremath{ \rho_{#1 #2}}}

%Goldpartikel
\newcommand{\npr}{\ensuremath{R}} %Nanopartikelradius
\newcommand{\cro}[1]{\ensuremath{\sigma_{\mathrm{#1}}}} % Wirkungsquerschnitt

%dielektr. Funktion
\newcommand{\epsi}[1]{\ensuremath{\varepsilon_{\mathrm{#1}}}}
\newcommand{\epsint}{\epsi{\mathcal{I}}}
\newcommand{\epsext}{\epsi{\mathcal{A}}}

% mathematische Konstrukte
% a function named #1 with argument #2; both in math
\newcommand{\func}[2]{\ensuremath{#1 \text{\small$\left(#2\right)$}}}

\renewcommand{\vec}[2][]{\ensuremath{\boldsymbol{#2}_{#1}}}
\renewcommand{\Re}{\mathfrak{Re}\kern-0.05em}
\renewcommand{\Im}{\mathfrak{Im}}

% Einheitsvektoren
\newcommand{\er}{\ensuremath{\vec[r]{\hat{e}}}}
\newcommand{\ex}{\ensuremath{\vec[x]{\hat{e}}}}
\newcommand{\ephi}{\ensuremath{\vec[\phi]{\hat{e}}}}
\newcommand{\etheta}{\ensuremath{\vec[\theta]{\hat{e}}}}

% em Felder und andere vektorielle Groessen

\newcommand{\evec}{\ensuremath{\vec{E}}}
\newcommand{\hvec}{\ensuremath{\vec{H}}}
\newcommand{\kvec}{\ensuremath{\vec{k}}}
\newcommand{\eovec}{\ensuremath{\vec[0]{E}}}
\newcommand{\escat}{\ensuremath{\vec[\mathrm{str}]{E}}}
\newcommand{\eint}{\ensuremath{\vec[\mathcal{I}]{E}}}
\newcommand{\einc}{\ensuremath{\vec[\mathrm{in}]{E}}}
\newcommand{\hscat}{\ensuremath{\vec[\mathrm{str}]{H}}}
\newcommand{\hint}{\ensuremath{\vec[\mathcal{I}]{H}}}
\newcommand{\hinc}{\ensuremath{\vec[\mathrm{in}]{H}}}
\newcommand{\bvec}{\ensuremath{\vec{B}}}
\newcommand{\rvec}{\ensuremath{\vec{r}}}
\newcommand{\rovec}{\ensuremath{\vec[0]{r}}}
\newcommand{\evecr}{\func{\evec}{\rvec}}
\newcommand{\hvecr}{\func{\hvec}{\rvec}}
\newcommand{\pvec}{\ensuremath{\vec{S}}}
\newcommand{\pvect}[1]{\ensuremath{\vec[\mathrm{#1}]{S}}}

% Mietheorie
\newcommand{\mmnj}{\ensuremath{\boldsymbol{M}^j_{mn}}}
\newcommand{\nmnj}{\ensuremath{\boldsymbol{N}^j_{mn}}}
\newcommand{\mmnbj}{\ensuremath{\boldsymbol{M}^{j\;\beta}_{mn}}}
\newcommand{\nmnbj}{\ensuremath{\boldsymbol{N}^{j\;\beta}_{mn}}}
\newcommand{\mm}[2]{\ensuremath{\boldsymbol{M}^{\,#1\, #2}_{mn}}}
\newcommand{\nn}[2]{\ensuremath{\boldsymbol{N}^{\,#1\, #2}_{mn}}}

\newcommand{\mmn}[1]{\ensuremath{\boldsymbol{M}^{#1}_{mn}}}
\newcommand{\nmn}[1]{\ensuremath{\boldsymbol{N}^{#1}_{mn}}}
\newcommand{\pimn}{\ensuremath{\pi_{mn}}}
\newcommand{\taumn}{\ensuremath{\tau_{mn}}}
\newcommand{\pmn}{\ensuremath{P^m_n}}
\newcommand{\emn}{\ensuremath{E_{mn}}}
\newcommand{\reln}{\ensuremath{\nu}}
\newcommand{\tilmn}{\ensuremath{{\tilde{m}\tilde{n}}}}

\newcommand{\lmax}{\ensuremath{\lambda_\mathrm{max}}}

\newcommand{\rvecp}[1][]{\ensuremath{\rvec{#1}'}}
\newcommand{\ri}[1]{\ensuremath{r_{#1}}}
\newcommand{\ucv}[1]{\ensuremath{\vec[#1]{e}}}
\newcommand{\magdip}{p}
%\newcommand{\xioffe}{x\kern-0.1em_I}
%\newcommand{\yioffe}{y\kern-0.1em_I}
%\newcommand{\zioffe}{z\kern-0.1em_I}


% the scalar product of #1 and #2
\newcommand{\scp}[2]{\ensuremath{#1 \cdot #2}}
% complex conjugate
\newcommand{\conj}[1]{\ensuremath{#1^{*}}}
% imaginary i
\newcommand{\imi}{\ensuremath{i}}
% upright d for integrals
\newcommand{\dint}{\text{d}}
% derivative d#1/d#2
\newcommand{\deriv}[2]{\ensuremath{\frac{\mathrm{d}#1}{\mathrm{d}#2}}}
\newcommand{\diff}[1]{\ensuremath{\frac{\mathrm{d}}{\mathrm{d}#1}}}
% exponential function as e^#1
\newcommand{\expe}[1]{\ensuremath{\text{e}^{#1}}}
% exponential function as exp(#1)
\renewcommand{\exp}[1]{\func{\text{exp}}{#1}}
% the big O for approximations
\newcommand{\bigo}[1]{\ensuremath{\func{\mathcal{O}}{#1}}}
% constant
\newcommand{\const}{\ensuremath{const}}
% the dirac delta "function"
\newcommand{\dirac}[1][\rvec]{\ensuremath{\func{\delta}{#1}}}
% a statistical average < #1 >
\newcommand{\statav}[1]{\langle #1\rangle}
% an equals sign with not too much space around it
\newcommand{\teq}{\kern-0.27em=\kern-0.1em} % tight equals
% the absolute value of something
\newcommand{\abs}[1]{\lvert#1\rvert}
% the vector norm
\newcommand{\vnorm}[1]{\ensuremath{\|#1\|}}
% the laplace operator
\newcommand{\lapl}{\triangle}
% the nabla operator
%\newcommand{\nabla}{\triangledown}
% the divergence operator
\DeclareMathOperator{\divergence}{div}

% QM
% dirac notation bra, ket, bracket, matrix element
\newcommand{\bra}[1]{\mbox{\ensuremath{\langle #1|}}}
\newcommand{\ket}[1]{\mbox{\ensuremath{| #1\rangle}}}
\newcommand{\braket}[2]{\mbox{\ensuremath{\langle #1| #2\rangle}}}
\newcommand{\mel}[3]{\ensuremath{\langle #2|#1| #3\rangle}}
% a function operator
\newcommand{\fop}[1][\rvec]{\widehat{\Psi}\text{\footnotesize$(#1, t)$}}
\newcommand{\fopi}[1]{\widehat{\Psi#1}\text{\footnotesize$(\rvec, t)$}}
\newcommand{\fopc}[1][\rvec]{\widehat{\Psi}^+\!\text{\footnotesize$(#1, t)$}}
% any quantum mechanical operator
\newcommand{\qop}[1]{\ensuremath{\widehat{#1}}}
% the bohr magneton
\newcommand{\mubohr}{\mu\kern-0.07em_\mathrm{B}}
\newcommand{\wf}[1][]{\ensuremath{\phi_\mathrm{#1}}}
\newcommand{\qnorm}[1]{\|#1\|_{\mathcal{L}^2}}


\newcommand{\konz}{\ensuremath{c_\mathrm{mol}}}
\newcommand{\csq}{\ensuremath{\chi^2}}

%Sonstige
\newcommand{\incflux}{\ensuremath{S_\mathrm{in}}}
\newcommand{\excflux}{\ensuremath{S_\mathrm{exc}}}
\newcommand{\dss}{\ensuremath{\tilde{d}}}
\newcommand{\registered}{\text{\textregistered}}
\newcommand{\gexc}{\ensuremath{g_{\mathrm{exc}}(\omega_\mathrm{exc})}}
\newcommand{\gem}{\ensuremath{g_\mathrm{em}(\omega)}}
\newcommand{\graman}{\ensuremath{g_\mathrm{EM}}}
\newcommand{\dip}{\ensuremath{\vec{p}}}
\newcommand{\rrD}{\ensuremath{\Gamma_\mathrm{r}}}
\newcommand{\rriD}{\ensuremath{\Gamma_{\mathrm{r}\, 0}}}
\newcommand{\retD}{\ensuremath{\Gamma_\mathrm{ET}}}
\newcommand{\rnr}{\ensuremath{\gamma_{\mathrm{nr}\,0}}}
%\newcommand{\mmu}[2]{\bvec{M}_{\mu\nu}^{#1\, #2}}
%\newcommand{\nnu}[2]{\bvec{N}_{\mu\nu}^{#1\, #2}}
%\newcommand{\ajmn}{a_{mn}^j}
%\newcommand{\bjmn}{b_{mn}^j}
%\newcommand{\pjlmn}{p_{mn}^{l}}
%\newcommand{\qjlmn}{q_{mn}^{l}}
\newcommand{\grw}{\ensuremath{g_\mathrm{r}(\omega)}}
\newcommand{\getw}{\ensuremath{g_\mathrm{ET}(\omega)}}
%\newcommand{\micro}{\ensuremath{\mu}}
